% Author: Nils Pukropp
% E-Mail: nils.pukropp@student.kit.edu
% License: MIT
\documentclass[aspectratio=169]{beamer}

\usepackage{fontspec}

\usetheme{metropolis}

\setmonofont[
  Contextuals={Alternate},
  Ligatures = TeX,
]{Fira Code}

\newcommand{\HUMONGOUS}{\Huge}

\usepackage{booktabs}
\usepackage[scale=2]{ccicons}

\usepackage{color}

\definecolor{linkcolor}{HTML}{EE0E61}

\hypersetup{
  colorlinks=true,
  linkcolor= {linkcolor},
  urlcolor= {linkcolor},
  bookmarks={true}
}

\definecolor{stringcolor}{RGB}{139,233,253}
\definecolor{commentcolor}{RGB}{189,147,249}
\definecolor{alertcolor}{RGB}{255,85,85}
\definecolor{keywordcolor}{RGB}{255 184 108}
\definecolor{FGround}{RGB}{248,248,242}
\definecolor{BGround}{RGB}{40,42,54}
\definecolor{classcolor}{HTML}{EE0E61}
\definecolor{numbercolor}{HTML}{6897bb}
\definecolor{darkgray}{HTML}{2b2b2b}
\definecolor{nicegreen}{HTML}{50fa7b}

\usepackage{listings}
\lstset{
  backgroundcolor=\color{BGround},
  language=Java,
  showspaces=false,
  showtabs=false,
  breaklines=true,
  columns = flexible,
  showstringspaces=true,
  breakatwhitespace=true,
  keywordstyle=\color{keywordcolor},
  morekeywords={enum},
  commentstyle=\color{commentcolor},
  stringstyle=\color{stringcolor},
  basicstyle=\footnotesize\ttfamily\color{FGround},
  numbers=left,
  numbersep=5pt,
  numberstyle=\color{numbercolor},
  captionpos=b,
  extendedchars=true,
  tabsize=2,
  escapeinside={\%*}{*)},
  moredelim=[is][\color{classcolor}]{@@c}{@@}
}

\usepackage{pgfplots}
\usepgfplotslibrary{dateplot}
\pgfplotsset{compat=1.17}

\usepackage{xspace}
\usepackage{multicol}

\title{\color{classcolor}Tutorium Woche 3 - Termin 1}
\subtitle{Übungsblatt 2}
\date{\today}
\author{Nils Pukropp}
\institute{INSTITUT FÜR PROGRAMMSTRUKTUREN UND DATENORGANISATION}
\titlegraphic{\hfill\includegraphics[height=1.5cm]{../logos/KIT-Logo.png}}
\metroset{block=fill}
\setbeamercolor{background canvas}{bg=darkgray}
\setbeamercolor{normal text}{fg=FGround, bg=darkgray}
\setbeamercolor{alerted text}{fg=alertcolor, bg=darkgray}
\setbeamercolor{example text}{fg=commentcolor, bg=darkgray}
\setbeamercolor{progressbar in foot}{fg=classcolor}
\setbeamercolor{frametitle}{bg=BGround, fg=FGround}


\begin{document}
\maketitle

\begin{frame}{Übersicht}
  \setbeamertemplate{section in toc}[sections numbered]
  \tableofcontents[hideallsubsections]
\end{frame}

\section{Vorstellung}
\begin{frame}[fragile]{Vorstellung}
  \begin{block}{Über mich}
  \begin{itemize}
    \item Nils Pukropp, 21 Jahre
    \begin{itemize}
      \item Informatik Bachelor im 4. Semester
    \end{itemize}
    \item E-Mail: \href{mailto:nils.pukropp@student.kit.edu}{nils.pukropp@student.kit.edu}
    \item Discord: \href{https://discord.gg/6GpaFE8w4y}{KIT Mathe Info}
  \end{itemize}
\end{block}
\end{frame}

\begin{frame}[fragile]{Vorstellung}
 \begin{columns}[T,onlytextwidth]
    \column{0.5\textwidth}
      \begin{block}{Regeln/Empfehlungen für MS Teams}
        \begin{itemize}
          \item Ton aus/Webcam an (falls, vorhanden, Alternative: droidcam)
          \item De-Anonymisierung aktiviert
          \item Fragen stellen \begin{itemize}
            \item einfach los fragen (einfach zwischen rein)
            \item oder Hand heben
            \item Duzen ist erwünscht
          \end{itemize}
        \end{itemize}
      \end{block}
    \column{0.05\textwidth}
    \column{0.5\textwidth}
      \includegraphics[scale=0.4]{../images/questions-meme.jpg}
  \end{columns}

\end{frame}

\section{Organisatorisches}
\begin{frame}[fragile]{Organisatorisches}
  \begin{alertblock}{Organisatorisches}
    \begin{itemize}
      \item Zum Vorrechen im Ilias anmelden
      \item Übungsblatt 3 \begin{itemize}
        \item \color{nicegreen}Ausgabe: \color{FGround} 17. Mai, 13 Uhr
        \item \color{numbercolor} Artemis: \color{FGround} 21. Mai, 13 Uhr
        \item \color{alertcolor}Deadline: \color{FGround} 01. Juni, 06 Uhr
      \end{itemize}
    \end{itemize}
  \end{alertblock}
  \begin{block}{Vorrechnen}
    Ihr werdet ziemlich sicher bestehen, wenn eure Abgabe selbstständig erstellt wurde und das Vorrechnen ist deutlich angenehmer als die Präsenzübung
  \end{block}
\end{frame}

\section{Übungsblatt 2}
\subsection{Aufgabe A: String-Utility}
\begin{frame}{Aufgabe A - Aufgabe und Lernziele}
  \begin{block}{Aufgabe}
    \pause
    Utility-Klasse für Strings schreiben
  \end{block}
  \pause
  \begin{block}{Lernziele}
    \begin{itemize}
      \pause
      \item Umgang mit Strings
      \pause
      \item Aufbau von Utility-Klassen
    \end{itemize}
  \end{block}
\end{frame}

\begin{frame}[fragile]{Aufgabe A - Privater Konstruktor}
  \begin{alertblock}{Privater Konstruktor und \color{keywordcolor}final}
    \begin{lstlisting}
public final class @@cStringUtility@@ {

    private @@cStringUtility@@() {
        throw new
            @@cIllegalStateException@@(ERROR_UTILITY_CLASS_INSTANTIATION);
    }
    \end{lstlisting}
  \end{alertblock}
\end{frame}

\begin{frame}[fragile]{Aufgabe A - reverse}
  \begin{block}{reverse}
    \begin{lstlisting}
public static @@cString@@ reverse(@@cString@@ word) {
    @@cString@@ reverse = "";

    // We need access to the indices of the characters,
    // so we have to use for
    // instead of foreach
    for (int i = word.length() - 1; i >= 0; i--) {
        reverse += word.charAt(i);
    }

    return reverse;
}
    \end{lstlisting}
  \end{block}
  \pause
  Alternative zu \color{classcolor}String \color{FGround} im Loop? \pause \color{classcolor}StringBuilder\color{FGround}!
\end{frame}

\begin{frame}[fragile]{Aufgabe A reverse mit StringBuilder}
  \begin{exampleblock}{reverse - StringBuilder}
    \begin{lstlisting}
public static @@cString@@ reverse(@@cString@@ word) {
    @@cStringBuilder@@ reverse = new @@cStringBuilder@@;

    for (int i = word.length() - 1; i >= 0; i--) {
        reverse.append(word.charAt(i));
    }

    return reverse.toString();
}
    \end{lstlisting}
  \end{exampleblock}
\end{frame}


\begin{frame}[fragile]{Aufgabe A - isPalindrome/removeCharacter}
  \begin{block}{isPalindrome}
    \begin{lstlisting}
public static boolean isPalindrome(@@cString@@ word) {
    return word.equals(reverse(word));
}
    \end{lstlisting}
  \end{block}
  \pause
  \begin{block}{removeCharacter}
    \begin{lstlisting}
public static @@cString@@ removeCharacter(@@cString@@ word, int index) {
    return word.substring(0, index) + word.substring(index + 1);
}
    \end{lstlisting}
  \end{block}
\end{frame}

\begin{frame}[fragile]{Aufgabe A - isAnagram}
  \begin{block}{isAnagram}
    \begin{lstlisting}
public static boolean isAnagram(@@cString@@ word1, @@cString@@ word2) {
    // If the words have a different lengths, they cannot be anagrams
    if (word1.length() != word2.length()) return false;
    @@cString@@ anagram = word2;
    // We do not need the indices so we use foreach
    for (final char character : word1.toCharArray()) {
        final int index = anagram.indexOf(character);
        if (index == -1) return false;
        anagram = removeCharacter(anagram, index);
    }
    return anagram.isEmpty();
}
    \end{lstlisting}
  \end{block}
\end{frame}

\begin{frame}[fragile]{Aufgabe A - capitalize}
  \begin{block}{capitalize}
    \begin{lstlisting}
public static @@cString@@ capitalize(@@cString@@ word) {
    return word.substring(0, 1).toUpperCase() + word.substring(1);
}
    \end{lstlisting}
  \end{block}
\end{frame}

\begin{frame}[fragile]{Aufgabe A - countCharacter}
  \begin{block}{countCharacter}
    \begin{lstlisting}
public static int countCharacter(@@cString@@ word, char character) {
    int count = 0;
    for (char wordCharacter : word.toCharArray()) {
        if (wordCharacter == character) {
            count += 1;
        }
    }
    return count;
}
    \end{lstlisting}
  \end{block}
\end{frame}




\subsection{Aufgabe B - Integer-Utility}

\begin{frame}{Aufgabe B - Aufgabe und Lernziele}
  \begin{block}{Aufgabe}
    \pause
    Utility-Klasse für Integer-Arrays
  \end{block}

  \pause
  \begin{block}{Lernziele}
    \begin{itemize}
      \item Aufbau von Utility-Klassen
      \pause
      \item Umgang mit Arrays
      \pause
      \item Umgang mit Schleifen \begin{itemize}
        \pause
        \item Wenn Index \color{alertcolor}nicht \color{FGround}benötigt wird: \color{keywordcolor}for-each\color{FGround}
        \pause
        \item Wenn Index benötigt wird: \color{keywordcolor}for\color{FGround}
        \pause
        \item Wenn wir eine Bedingung haben: \color{keywordcolor}while\color{FGround}
      \end{itemize}
    \end{itemize}
  \end{block}
\end{frame}

\begin{frame}[fragile]{Wann welcher Schleifentyp?}
  \begin{exampleblock}{For-each}
    \begin{lstlisting}
public static void printoutArray(int[] arr) {
    for (final int num : arr) {
        @@cSystem@@.out.println(@@cString@@.valueOf(num));
    }
}
    \end{lstlisting}
  \end{exampleblock}
\end{frame}

\begin{frame}[fragile]{Wann welcher Schleifentyp?}
  \begin{exampleblock}{For}
    \begin{lstlisting}
public static int indexOf(int[] arr, final int element) {
    for (int i = 0; i < arr.length; i++) {
        if ( arr[i] == element ) return i;
    }
    return -1;
}
    \end{lstlisting}
  \end{exampleblock}
\end{frame}

\begin{frame}[fragile]{Wann welcher Schleifentyp?}
  \begin{exampleblock}{While}
    \begin{lstlisting}
...
// integerList ist von Typ List<Integer>
@@cIterator@@<@@cInteger@@> iter = integerList.iterator();
while (iter.hasNext()) {
    @@cSystem@@.out.println(iter.next());
}
...
    \end{lstlisting}
  \end{exampleblock}
\end{frame}

\begin{frame}[fragile]{Aufgabe B - Privater Konstruktor und final}
  \begin{alertblock}{Privater Konstruktor und \color{keywordcolor}final}
    \begin{lstlisting}
public final class @@cIntegerUtility@@ {
    /**
     * Creates a new instance of {@link IntegerUtility}.
     *
     * @throws IllegalStateException if constructor is called
     * because {@link IntegerUtility} is a utility class.
     */
    private @@cIntegerUtility@@() {
        throw new
            @@cIllegalStateException@@(ERROR_UTILITY_CLASS_INSTANTIATION);
    }
    \end{lstlisting}
  \end{alertblock}
\end{frame}

\begin{frame}[fragile]{Aufgabe B - Hilfsmethoden}
  \begin{block}{truncate - Hilfsmethode}
    \begin{lstlisting}
private static int[] truncate(int[] values, int length) {
    // Check whether array is long enough for requested truncation.
    if (length > values.length) {
        throw new
            @@cIllegalArgumentException@@(ERROR_INSUFFICIENT_ARRAY_LENGTH);
    }
    // Truncate array to number of written values and return it.
    int[] truncatedValues = new int[length];
    @@cSystem@@.arraycopy(values, 0, truncatedValues, 0, length);
    return truncatedValues;
}
    \end{lstlisting}
  \end{block}
\end{frame}

\begin{frame}[fragile]{Aufgabe B - Hilfsmethoden}
  \begin{block}{isOdd - Hilfsmethode}
    \begin{lstlisting}
private static boolean isOdd(int value) {
    return @@cMath@@.abs(value % 2) == 1;
}
    \end{lstlisting}
  \end{block}

  \pause

  \begin{block}{contains - Hilfsmethode}
    \begin{lstlisting}
private static boolean contains(int[] values, int value) {
    return search(values, value) != RETURN_VALUE_NOT_FOUND;
}
    \end{lstlisting}
  \end{block}
\end{frame}

\begin{frame}[fragile]{Aufgabe B - greaterThan}
  \begin{block}{greaterThan}
    \begin{lstlisting}
public static int[] greaterThan(int[] values, int threshold) {
    // Initialize and fill new array with values greater than threshold
    int[] greaterValues = new int[values.length];
    int indexOfLastValueAdded = -1;
    for (int value : values) {
        // If a greater value is found, write it to next position in array
        if (value > threshold) {
            greaterValues[++indexOfLastValueAdded] = value;
        }
    }
    // Truncate result to number of added elements and return it
    return truncate(greaterValues, indexOfLastValueAdded + 1);
}
    \end{lstlisting}
  \end{block}
\end{frame}


\begin{frame}[fragile]{Aufgabe B - getOdds}
  \begin{block}{getOdds}
    \begin{lstlisting}
public static int[] getOdds(int[] values) {
    // Initialize and fill new array with odd values
    int[] oddValues = new int[values.length];
    int indexOfLastValueAdded = -1;
    for (int value : values) {
        // If an odd value is found, write it to next position in array
        if (isOdd(value)) {
            oddValues[++indexOfLastValueAdded] = value;
        }
    }
    // Truncate result to number of added elements and return it
    return truncate(oddValues, indexOfLastValueAdded + 1);
}
    \end{lstlisting}
  \end{block}
\end{frame}



\begin{frame}[fragile]{Aufgabe B - search}
  \begin{block}{search}
    \begin{lstlisting}
public static int search(int[] values, int key) {
    for (int i = 0; i < values.length; i++) {
        if (values[i] == key) {
            return i;
        }
    }
    return RETURN_VALUE_NOT_FOUND;
}
    \end{lstlisting}
  \end{block}
\end{frame}

\begin{frame}[fragile]{Aufgabe B - sort}
  \begin{block}{sort}
    \begin{lstlisting}
public static int[] sort(int[] values) {
    // Copy values to new array
    int[] sortedValues = new int[values.length];
    @@cSystem@@.arraycopy(values, 0, sortedValues, 0, values.length);
    // Sort copied array with bubblesort
    for (int i = sortedValues.length; i > 1; i--) {
        ...
        ...
    }
    return sortedValues;
}
    \end{lstlisting}
  \end{block}
\end{frame}

\begin{frame}[fragile]{Aufgabe B - sort}
  \begin{block}{sort}
    \begin{lstlisting}
...
for (int j = 0; j < i - 1; j++) {
    if (sortedValues[j] > sortedValues[j + 1]) {
        int oldHighIndexValue = sortedValues[j + 1];
        sortedValues[j + 1] = sortedValues[j];
        sortedValues[j] = oldHighIndexValue;
    }
}
...
    \end{lstlisting}
  \end{block}
\end{frame}

\begin{frame}[fragile]{Aufgabe B - median}
  \begin{block}{median}
    \begin{lstlisting}
public static double median(int[] values) {
    // Sort array and calculate its (high) middle index.
    int[] sortedArray = sort(values);
    int flooredHalfLength = sortedArray.length / 2;
    // Calculate median
    double median;
    if (isOdd(sortedArray.length)) {
        median = sortedArray[flooredHalfLength];
    } else {
        int topHalfValue = sortedArray[flooredHalfLength];
        int bottomHalfValue = sortedArray[flooredHalfLength - 1];
        median = ((double) topHalfValue + (double) bottomHalfValue) / 2d;
    }
    return median;
}
    \end{lstlisting}
  \end{block}
\end{frame}

\begin{frame}[fragile]{Aufgabe B - common}
  \begin{block}{common}
    \begin{lstlisting}
public static int[] common(int[] a, int[] b) {
    int[] commonValues = new int[0];
    for (int value : a) {
        // Check whether value can be found in the other array but was
        // not added to common values yet
        if (contains(b, value) && !contains(commonValues, value)) {
            ...
            ...
        }
    }
    return commonValues;
}
    \end{lstlisting}
  \end{block}
\end{frame}

\begin{frame}[fragile]{Aufgabe B - common}
  \begin{block}{common}
    \begin{lstlisting}
...
// Create a temporary lengthened array based on the already
// identified common values
int[] tmpCommonValues = new int[commonValues.length + 1];
@@cSystem@@.arraycopy(commonValues, 0, tmpCommonValues, 0,
    commonValues.length);
// Add new common value to temporary array and write temporary
// array back to actual array
tmpCommonValues[tmpCommonValues.length - 1] = value;
commonValues = tmpCommonValues;
...
    \end{lstlisting}
  \end{block}
\end{frame}

\subsection{Aufgabe C - Kalender}

\begin{frame}{Aufgabe C - Kalender}
  \begin{block}{Aufgabe}
    \pause
    Einen Kalender in OOP modellieren
  \end{block}

  \pause
  \begin{block}{Lernziele}
    \begin{itemize}
      \item OO-Programmierung
      \pause
      \item Passende Typenwahl für Attribute/Objekte
      \pause
      \item Grundlegender Aufbau einer Klasse
    \end{itemize}
  \end{block}
\end{frame}

\begin{frame}[fragile]{Aufgabe C - Weekday}
  \begin{block}{Weekday}
    \begin{lstlisting}
/**
 * Weekday is an enum representing the seven days of the week.
 *
 * @author Moritz Gstuer
 * @version 1.0.0
 */
public enum @@cWeekday@@ {
    /**
     * An instance representing the day of the week called Monday.
     */
    MONDAY,
    ...
    TUESDAY, WEDNESDAY, THURSDAY, FRIDAY, SATURDAY, SUNDAY;
}
    \end{lstlisting}
  \end{block}
\end{frame}

\begin{frame}[fragile]{Aufgabe C - Month}
  \begin{block}{Month}
    \begin{lstlisting}
/**
 * Month is an enum representing the 12 months of a year.
 *
 * @author Moritz Gstuer
 * @version 1.0.0
 */
public enum @@cMonth@@ {
    /**
     * An instance representing the first month of a year.
     */
    JANUARY,
    ...
    FEBRUARY, MARCH, APRIL, MAY, JUNE, JULY, AUGUST,
    SEPTEMBER, OCTOBER, NOVEMBER, DECEMBER;
}
    \end{lstlisting}
  \end{block}
\end{frame}

\begin{frame}[fragile]{Aufgabe C - Time}
  \begin{block}{Time}
    \begin{lstlisting}
/**
 * Represents a specific point of time.
 *
 * @author Moritz Gstuer
 * @version 1.0.0
 */
public class @@cTime@@ {
    private int hour;
    private int minute;
    private int second;
}
    \end{lstlisting}
  \end{block}
\end{frame}

\begin{frame}[fragile]{Aufgabe C - Date}
  \begin{block}{Date}
    \begin{lstlisting}
/**
 * Represents a calendar date of the Gregorian calendar.
 *
 * @author Moritz Gstuer
 * @version 1.0.0
 */
public class @@cDate@@ {
    private int year;
    private @@cMonth@@ month;
    private int day;
}
    \end{lstlisting}
  \end{block}
\end{frame}

\begin{frame}[fragile]{Aufgabe C - DateTime}
  \begin{block}{DateTime}
    \begin{lstlisting}
/**
 * Represents the combination of a calendar date and a
 * specific point of time.
 *
 * @author Moritz Gstuer
 * @version 1.0.0
 */
public class @@cDateTime@@ {
    private @@cDate@@ date;
    private @@cTime@@ time;
}
    \end{lstlisting}
  \end{block}
\end{frame}

\begin{frame}[fragile]{Aufgabe C - Timezone}
  \begin{block}{Timezone}
    \begin{lstlisting}
/**
 * Represents a zone of time according to the ISO 8601 standard.
 *
 * @author Moritz Gstuer
 * @version 1.0.0
 */
public class @@cTimezone@@ {
    private @@cString@@ identifier;
    private int timeOffset;
}
    \end{lstlisting}
  \end{block}
\end{frame}

\begin{frame}[fragile]{Aufgabe C - ZonedDateTime}
  \begin{block}{ZonedDateTime}
    \begin{lstlisting}
/**
 * Represents a point of time and date within a specific timezone.
 *
 * @author Moritz Gstuer
 * @version 1.0.0
 */
public class @@cZonedDateTime@@ {
    private @@cDateTime@@ dateTime;
    private @@cTimezone@@ timezone;
}
    \end{lstlisting}
  \end{block}
\end{frame}

\section{Korrektur}
\begin{frame}{Korrektur - Disclaimer}
  \begin{alertblock}{Disclaimer}
    Ich habe eure Abgaben nicht zwingend korrigiert, also nutzt das Beschwerde-System von Artemis!
  \end{alertblock}

  \begin{block}{Ich beantworte gerne, ...}
    \begin{itemize}
      \item Inhaltliche Fragen
      \item Fragen übers Studium
      \item Informatik-Fragen
    \end{itemize}
  \end{block}
\end{frame}

\begin{frame}{Korrektur - Häufige Fehler}
  \begin{block}{Häufige Fehler}
    \begin{itemize}
      \pause
      \item ordentlicher Javadoc (\color{alertcolor}ab Blatt 3 gibts Abzug\color{FGround})
      \pause
      \item Quelltext-Duplikate durch Hilfsmethoden vermeiden
      \pause
      \item Utility-Klassen brauchen einen privaten Konstruktor und müssen \color{keywordcolor}final \color{FGround}sein
      \pause
      \item Java-Sourcecode sollte in einer Datei mit der Dateiendung .java liegen
      \pause
      \item Schaut was Artemis euch ausgibt. Wenn eure Abgabe nicht Kompiliert gibt es 0 Punkte!
    \end{itemize}
  \end{block}
\end{frame}

\subsection{How to Javadoc}

\begin{frame}[fragile]{Javadoc}
  \begin{exampleblock}{Aufbau eines Methoden-Javadoc}
    \begin{lstlisting}
/**
 * Kurze Beschreibung was die Methode macht.
 * Im besten Fall wird auch erwähnt welche Randbedingungen auftreten
 * können.
 *
 * @throws EXCEPION was und wann eine Methode EXCEPION wirft
 * @param PARAMETER_NAME beschreibt einen Parameter der Methode.
 * @return beschreibt was die Methode zurückgibt.
 */
    \end{lstlisting}
  \end{exampleblock}
\end{frame}

\begin{frame}[fragile]{Javadoc}
  \begin{exampleblock}{Aufbau eines Klassen-Javadoc}
    \begin{lstlisting}
/**
 * Kurze Beschreibung was die Klasse modelliert.
 * Mit {@link path.to.some.Clazz} kann man andere
 * Klassen verlinken oder mit {@link Clazz#Methode/Variabel}
 * können auch Attribute, Variabeln und Methoden referenziert werden
 *
 * @author u-kürzel
 * @version 1.0
 */
    \end{lstlisting}
  \end{exampleblock}
\end{frame}

\subsection{Tipps Blatt 3}

\begin{frame}{Tipps zu Blatt 3}
  \begin{block}{Was ihr beachten solltet}
    \begin{itemize}
      \pause
      \item Achtet auf eine saubere Trennung von Logik und Benutzerinteraktion
      \pause
      \item Dafür bietet es sich an die Musterlösungen vom vergangenen Semester anzuschauen
      \pause
      \item Referenziert Objekte nicht über Strings!
      \pause
      \item Schreibt sinnvollen Javadoc
      \pause
      \item Vermeidet in Programmieren statische Variabeln, die sich wie Attribute verhalten. \pause \linebreak
        Mann kann sich merken:
        \begin{itemize}
          \pause
          \item In Programmieren ist eigentlich alles \color{keywordcolor}static \color{FGround}auch \color{keywordcolor}final\color{FGround}
          \pause
          \item Statt einen statischen Beobachter für Objekte zu haben, sollte dieser Beobachter evt. ein Attribut von einem anderen Objekt sein.
        \end{itemize}
    \end{itemize}
  \end{block}
\end{frame}



\section{Fragen und Vorrechnen}

\begin{frame}
  \begin{center}\HUMONGOUS Fragen?\end{center}
\end{frame}

\begin{frame}
  \begin{center}\HUMONGOUS Vorrechnen!\end{center}
\end{frame}

\end{document}
