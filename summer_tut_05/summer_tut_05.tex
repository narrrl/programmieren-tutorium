% Author: Nils Pukropp
% E-Mail: nils.pukropp@student.kit.edu
% License: MIT
\documentclass[aspectratio=169]{beamer}

\usepackage{fontspec}

\usetheme{metropolis}

\setmonofont[
  Contextuals={Alternate},
  Ligatures = TeX,
]{Fira Code}

\newcommand{\HUMONGOUS}{\Huge}

\usepackage{booktabs}
\usepackage[scale=2]{ccicons}

\usepackage{color}

\definecolor{linkcolor}{HTML}{EE0E61}

\hypersetup{
  colorlinks=true,
  linkcolor= {linkcolor},
  urlcolor= {linkcolor},
  bookmarks={true}
}

\definecolor{stringcolor}{RGB}{139,233,253}
\definecolor{commentcolor}{RGB}{189,147,249}
\definecolor{alertcolor}{RGB}{255,85,85}
\definecolor{keywordcolor}{RGB}{255 184 108}
\definecolor{FGround}{RGB}{248,248,242}
\definecolor{BGround}{RGB}{40,42,54}
\definecolor{classcolor}{HTML}{EE0E61}
\definecolor{numbercolor}{HTML}{6897bb}
\definecolor{darkgray}{HTML}{2b2b2b}
\definecolor{nicegreen}{HTML}{50fa7b}

\usepackage{listings}
\lstset{
  backgroundcolor=\color{BGround},
  language=Java,
  showspaces=false,
  showtabs=false,
  breaklines=true,
  columns = flexible,
  showstringspaces=true,
  breakatwhitespace=true,
  keywordstyle=\color{keywordcolor},
  morekeywords={enum},
  commentstyle=\color{commentcolor},
  stringstyle=\color{stringcolor},
  basicstyle=\footnotesize\ttfamily\color{FGround},
  numbers=left,
  numbersep=5pt,
  numberstyle=\color{numbercolor},
  captionpos=b,
  extendedchars=true,
  tabsize=2,
  escapeinside={\%*}{*)},
  moredelim=[is][\color{classcolor}]{@@c}{@@}
}

\usepackage{pgfplots}
\usepgfplotslibrary{dateplot}
\pgfplotsset{compat=1.16}

\usepackage{xspace}
\usepackage{multicol}

\title{\color{classcolor}Tutorium Woche 3 - Termin 1}
\subtitle{Übungsblatt 2}
\date{\today}
\author{Nils Pukropp}
\institute{INSTITUT FÜR PROGRAMMSTRUKTUREN UND DATENORGANISATION}
\titlegraphic{\hfill\includegraphics[height=1.5cm]{../logos/KIT-Logo.png}}
\metroset{block=fill}
\setbeamercolor{background canvas}{bg=darkgray}
\setbeamercolor{normal text}{fg=FGround, bg=darkgray}
\setbeamercolor{alerted text}{fg=alertcolor, bg=darkgray}
\setbeamercolor{example text}{fg=commentcolor, bg=darkgray}
\setbeamercolor{progressbar in foot}{fg=classcolor}
\setbeamercolor{frametitle}{bg=BGround, fg=FGround}


\begin{document}
\maketitle

\begin{frame}{Übersicht}
  \setbeamertemplate{section in toc}[sections numbered]
  \tableofcontents[hideallsubsections]
\end{frame}

\section{Vorstellung}
\begin{frame}[fragile]{Vorstellung}
  \begin{block}{Über mich}
  \begin{itemize}
    \item Nils Pukropp, 21 Jahre
    \begin{itemize}
      \item Informatik Bachelor im 4. Semester
    \end{itemize}
    \item E-Mail: \href{mailto:nils.pukropp@student.kit.edu}{nils.pukropp@student.kit.edu}
    \item Discord: \href{https://discord.gg/6GpaFE8w4y}{KIT Mathe Info}
  \end{itemize}
\end{block}
\end{frame}

\begin{frame}[fragile]{Vorstellung}
 \begin{columns}[T,onlytextwidth]
    \column{0.5\textwidth}
      \begin{block}{Regeln/Empfehlungen für MS Teams}
        \begin{itemize}
          \item Ton aus/Webcam an (falls, vorhanden, Alternative: droidcam)
          \item De-Anonymisierung aktiviert
          \item Fragen stellen \begin{itemize}
            \item einfach los fragen (einfach zwischen rein)
            \item oder Hand heben
            \item Duzen ist erwünscht
          \end{itemize}
        \end{itemize}
      \end{block}
    \column{0.05\textwidth}
    \column{0.5\textwidth}
      \includegraphics[scale=0.4]{../images/questions-meme.jpg}
  \end{columns}

\end{frame}

\section{Organisatorisches}
\begin{frame}[fragile]{Organisatorisches}
  \begin{alertblock}{Organisatorisches}
    \begin{itemize}
      \item Schauen ob man den Übungsschein hat
      \item Abschlussaufgabe 1 \begin{itemize}
        \item \color{nicegreen}Ausgabe: \color{FGround} 12. Juli, 13 Uhr
        \item \color{numbercolor} Artemis: \color{FGround} 26. Juli, 13 Uhr
        \item \color{alertcolor}Deadline: \color{FGround} 08. August, 06 Uhr
      \end{itemize}
      \item Abschlussaufgabe 2 \begin{itemize}
        \item \color{nicegreen}Ausgabe: \color{FGround} 26. Juli, 13 Uhr
        \item \color{numbercolor} Artemis: \color{FGround} 09. August, 13 Uhr
        \item \color{alertcolor}Deadline: \color{FGround} 24. August, 06 Uhr
      \end{itemize}
    \end{itemize}
  \end{alertblock}
  \begin{block}{Abschlussaufgaben}
    Nehmt euch genügend Zeit für die Abschlussaufgaben, also fangt nicht erst ein paar Tage vor Abgabe an.
  \end{block}
\end{frame}

\section{Interaktive Benutzerschnittstellen}
\subsection{Enum als Benutzerschnittstelle}
\begin{frame}{Enum als Benutzerschnittstelle}
  \begin{block}{Vorprogrammieren in VSCode}
    Lösung wird im Ilias hochgeladen! Ihr könnt einfach zuschauen und Fragen stellen
  \end{block}
\end{frame}

\subsection{Beispiel Musterlösung Blatt 5.}
\begin{frame}{Beispiel Musterlösung Blatt 5.}
  Siehe Musterlösung
\end{frame}

\subsection{Alternative: Command-Pattern}
\begin{frame}{Alternative: Command-Pattern}
  \begin{block}{Vorprogrammieren in VSCode}
    Lösung wird im Ilias hochgeladen! Ihr könnt einfach zuschauen und Fragen stellen
  \end{block}
\end{frame}

\section{Tipps für die Abschlussaufgaben}

\subsection{Namenskonventionen}
\begin{frame}[fragile]{Namenskonventionen}
  \begin{block}{Attribut- und Methodennamen}
    lowerCamelCase \linebreak
    \color{nicegreen}Do.\color{FGround}
    \begin{lstlisting}[numbers=none]
private int turnCount;
    \end{lstlisting}
    \color{alertcolor}Don't.\color{FGround}
    \begin{lstlisting}[numbers=none]
private int n;
    \end{lstlisting}
  \end{block}
\end{frame}

% TODO: add tipps for finals



\begin{frame}
  \begin{center}\HUMONGOUS Fragen?
    \pause
    \linebreak
    \linebreak
    Viel Erfolg!
  \end{center}
\end{frame}
\end{document}
