\documentclass[aspectratio=169]{beamer}

\usepackage{fontspec}

\usetheme{metropolis}

\setmonofont[
  Contextuals={Alternate},
  Ligatures = TeX,
]{Fira Code}


\usepackage{booktabs}
\usepackage[scale=2]{ccicons}

\usepackage{color}

\definecolor{linkcolor}{HTML}{EE0E61}

\hypersetup{
  colorlinks=true,
  linkcolor= {linkcolor},
  urlcolor= {linkcolor},
  bookmarks={true}
}

\definecolor{stringcolor}{RGB}{139,233,253}
\definecolor{commentcolor}{RGB}{189,147,249}
\definecolor{alertcolor}{RGB}{255,85,85}
\definecolor{keywordcolor}{RGB}{255 184 108}
\definecolor{FGround}{RGB}{248,248,242}
\definecolor{BGround}{RGB}{40,42,54}
\definecolor{classcolor}{HTML}{EE0E61}
\definecolor{numbercolor}{HTML}{6897bb}
\definecolor{darkgray}{HTML}{2b2b2b}
\definecolor{nicegreen}{HTML}{50fa7b}

\usepackage{listings}
\lstset{
  backgroundcolor=\color{BGround},
  language=Java,
  showspaces=false,
  showtabs=false,
  breaklines=true,
  columns = flexible,
  showstringspaces=true,
  breakatwhitespace=true,
  keywordstyle=\color{keywordcolor},
  commentstyle=\color{commentcolor},
  stringstyle=\color{stringcolor},
  basicstyle=\footnotesize\ttfamily\color{FGround},
  numbers=left,
  numbersep=5pt,
  numberstyle=\color{numbercolor},
  captionpos=b,
  extendedchars=true,
  tabsize=2,
  escapeinside={\%*}{*)},
  moredelim=[is][\color{classcolor}]{@@c}{@@}
}

\usepackage{pgfplots}
\usepgfplotslibrary{dateplot}
\pgfplotsset{compat=1.17}

\usepackage{xspace}
\usepackage{multicol}

\title{\color{classcolor}Tutorium Woche 3 - Termin 1}
\subtitle{Übungsblatt 2}
\date{\today}
\author{Nils Pukropp}
\institute{INSTITUT FÜR PROGRAMMSTRUKTUREN UND DATENORGANISATION}
\titlegraphic{\hfill\includegraphics[height=1.5cm]{logos/KIT-Logo.png}}
\metroset{block=fill}
\setbeamercolor{background canvas}{bg=darkgray}
\setbeamercolor{normal text}{fg=FGround, bg=darkgray}
\setbeamercolor{alerted text}{fg=alertcolor, bg=darkgray}
\setbeamercolor{example text}{fg=commentcolor, bg=darkgray}
\setbeamercolor{progressbar in foot}{fg=classcolor}
\setbeamercolor{frametitle}{bg=BGround, fg=FGround}


\begin{document}
\maketitle

\begin{frame}{Übersicht}
  \setbeamertemplate{section in toc}[sections numbered]
  \tableofcontents[hideallsubsections]
\end{frame}

\section{Vorstellung}
\begin{frame}[fragile]{Vorstellung}
  \begin{block}{Über mich}
  \begin{itemize}
    \item Nils Pukropp, 21 Jahre
    \begin{itemize}
      \item Informatik Bachelor im 4. Semester
    \end{itemize}
    \item E-Mail: \href{mailto:nils.pukropp@student.kit.edu}{nils.pukropp@student.kit.edu}
  \end{itemize}
\end{block}
\end{frame}

\begin{frame}[fragile]{Vorstellung}
 \begin{columns}[T,onlytextwidth]
    \column{0.5\textwidth}
      \begin{block}{Regeln/Empfehlungen für MS Teams}
        \begin{itemize}
          \item Ton aus/Webcam an (falls, vorhanden, Alternative: droidcam)
          \item De-Anonymisierung aktiviert
          \item Fragen stellen \begin{itemize}
            \item einfach los fragen (einfach zwischen rein)
            \item oder Hand heben
            \item Duzen ist erwünscht
          \end{itemize}
        \end{itemize}
      \end{block}
    \column{0.05\textwidth}
    \column{0.5\textwidth}
      \includegraphics[scale=0.4]{../images/questions-meme.jpg}
  \end{columns}

\end{frame}

\section{Organisatorisches}
\begin{frame}[fragile]{Organisatorisches}
  \begin{alertblock}{Organisatorisches}
    \begin{itemize}
      \item Zum Vorrechen im Ilias anmelden
      \item Übungsblatt 3 \begin{itemize}
        \item \color{nicegreen}Ausgabe: \color{FGround} 17. Mai, 13 Uhr
        \item \color{numbercolor} Artemis: \color{FGround} 21. Mai, 13 Uhr
        \item \color{alertcolor}Deadline: \color{FGround} 01. Juni, 06 Uhr
      \end{itemize}
    \end{itemize}
  \end{alertblock}
  \begin{block}{Vorrechnen}
    Ihr werdet ziemlich sicher bestehen, wenn eure Abgabe selbstständig erstellt wurde und das Vorrechnen ist deutlich angenehmer als die Präsenzübung
  \end{block}
\end{frame}

\section{Übungsblatt 2}
\subsection{Aufgabe A: String-Utility}
\begin{frame}{Aufgabe A}
  \begin{block}{Aufgabe}
    \pause
    Utility-Klasse für Strings schreiben
  \end{block}
  \pause
  \begin{block}{Lernziele}
    \begin{itemize}
      \pause
      \item Umgang mit Strings
      \pause
      \item Aufbau von Utility-Klassen
    \end{itemize}
  \end{block}
\end{frame}

\begin{frame}[fragile]{Aufgabe A}
  \begin{alertblock}{Privater Konstruktur und final}
    \begin{lstlisting}
public final class @@cStringUtility@@ {

    private @@cStringUtility@@() {
        throw new
            @@cIllegalStateException@@(ERROR_UTILITY_CLASS_INSTANTIATION);
    }
    \end{lstlisting}
  \end{alertblock}
  \pause
  \begin{block}{capitalize}
    \begin{lstlisting}
public static @@cString@@ capitalize(@@cString@@ word) {
    return word.substring(0, 1).toUpperCase() + word.substring(1);
}
    \end{lstlisting}
  \end{block}
\end{frame}

\begin{frame}[fragile]{Aufgabe A}
  \begin{block}{countCharacter}
    \begin{lstlisting}
public static int countCharacter(@@cString@@ word, char character) {
    int count = 0;
    for (char wordCharacter : word.toCharArray()) {
        if (wordCharacter == character) {
            count += 1;
        }
    }
    return count;
}
    \end{lstlisting}
  \end{block}
\end{frame}

\begin{frame}[fragile]{Aufgabe A}
  \begin{block}{isAnagram}
    \begin{lstlisting}
public static boolean isAnagram(@@cString@@ word1, @@cString@@ word2) {
    // If the words have a different lengths, they cannot be anagrams
    if (word1.length() != word2.length()) return false;
    @@cString@@ anagram = word2;
    // We do not need the indices so we use foreach
    for (final char character : word1.toCharArray()) {
        final int index = anagram.indexOf(character);
        if (index == -1) return false;
        anagram = removeCharacter(anagram, index);
    }
    return anagram.isEmpty();
}
    \end{lstlisting}
  \end{block}
\end{frame}

\begin{frame}[fragile]{Aufgabe A}
  \begin{block}{isPalindrome}
    \begin{lstlisting}
public static boolean isPalindrome(@@cString@@ word) {
    return word.equals(reverse(word));
}
    \end{lstlisting}
  \end{block}
  \pause
  \begin{block}{removeCharacter}
    \begin{lstlisting}
public static @@cString@@ removeCharacter(@@cString@@ word, int index) {
    return word.substring(0, index) + word.substring(index + 1);
}
    \end{lstlisting}
  \end{block}
\end{frame}

\begin{frame}[fragile]{Aufgabe A}
  \begin{block}{reverse}
    \begin{lstlisting}
public static @@cString@@ reverse(@@cString@@ word) {
    @@cString@@ reverse = "";

    // We need access to the indices of the characters,
    // so we have to use for
    // instead of foreach
    for (int i = word.length() - 1; i >= 0; i--) {
        reverse += word.charAt(i);
    }

    return reverse;
}
    \end{lstlisting}
  \end{block}
  \pause
  Alternative zu \color{classcolor}String \color{FGround} im Loop? \pause \color{classcolor}Stringbuilder\color{FGround}!
\end{frame}

\begin{frame}[fragile]{Aufgabe A}
  \begin{exampleblock}{reverse - StringBuilder}
    \begin{lstlisting}
public static @@cString@@ reverse(@@cString@@ word) {
    @@cStringBuilder@@ reverse = new @@cStringBuilder@@;

    for (int i = word.length() - 1; i >= 0; i--) {
        reverse.append(word.charAt(i));
    }

    return reverse.toString();
}
    \end{lstlisting}
  \end{exampleblock}
\end{frame}

\subsection{Aufgabe B - Integer-Utility}

\begin{frame}{Aufgabe B}
  \begin{block}{Aufgabe}
    \pause
    Utility-Klasse für Integer-Arrays
  \end{block}

  \pause
  \begin{block}{Lernziele}
    \begin{itemize}
      \item Aufbau von Utility-Klassen
      \pause
      \item Umgang mit Arrays
      \pause
      \item Umgang mit Schleifen \begin{itemize}
        \pause
        \item Wenn Index \color{alertcolor}nicht \color{FGround}benötigt wird: \color{keywordcolor}for-each\color{FGround}
        \pause
        \item Wenn Index benötigt wird: \color{keywordcolor}for\color{FGround}
        \pause
        \item Wenn wir eine Bedingung haben: \color{keywordcolor}while\color{FGround}
      \end{itemize}
    \end{itemize}
  \end{block}
\end{frame}

\begin{frame}[fragile]{Wann welcher Schleifentyp?}
  \begin{exampleblock}{For-each}
    \begin{lstlisting}
public static void printoutArray(int[] arr) {
    for (final int num : arr) {
      @@cSystem@@.out.println(@@cString@@.valueOf(num));
    }
}
    \end{lstlisting}
  \end{exampleblock}
\end{frame}

\begin{frame}[fragile]{Wann welcher Schleifentyp?}
  \begin{exampleblock}{For}
    \begin{lstlisting}
public static int indexOf(int[] arr, final int element) {
    for (int i = 0; i < arr.length; i++) {
      if ( arr[i] == element ) return i;
    }
    return -1;
}
    \end{lstlisting}
  \end{exampleblock}
\end{frame}

\begin{frame}[fragile]{Wann welcher Schleifentyp?}
  \begin{exampleblock}{While}
    \begin{lstlisting}
...
// integerList ist von Typ List<Integer>
@@cIterator@@<@@cInteger@@> iter = integerList.iterator();
while (iter.hasNext()) {
    @@cSystem@@.out.println(iter.next());
}
...
    \end{lstlisting}
  \end{exampleblock}
\end{frame}

\section{Fragen und Vorrechnen}

\begin{frame}
  \begin{center}\LARGE Fragen?\end{center}
\end{frame}

\begin{frame}
  \begin{center}\LARGE Vorrechnen!\end{center}
\end{frame}

\end{document}
